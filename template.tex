\documentclass{invoice}

\def \tab {\hspace*{3ex}}

\begin{document}

\hfill{\Huge\bf Invoice \#<%= @ inv-id %>}\hfill
\bigskip\break
{\large from \bf <%= entity-name (@ issuer) %>}
\hfill issued <%= invoice-date (@ inv) %> \\
\hrule

<%= entity-address (@ issuer) %>
\hfill <%= entity-phone (@ issuer) %> \\
\tab Tax id: <%= entity-tax-id (@ issuer) %> \\
\tab IBAN: <%= entity-iban (@ issuer) %> \\
\tab SWIFT: <%= entity-swift (@ issuer) %> \\
\tab Legal form: <%= entity-legal-form (@ issuer) %> \\
\\ \\
{\bf Invoice To:} \\
\tab <%= entity-name (@ payer) %> \\
\tab <%= entity-address (@ payer) %> \\
\tab Id: <%= entity-id (@ payer) %> \\
\tab Vat id: <%= entity-vat-id (@ payer) %> \\

{\bf Due Date:} \\
\tab <%= invoice-due-date (@ inv) %> \\

%----------------------------------------------------------------------------------------
%	TABLE OF EXPENSES
%----------------------------------------------------------------------------------------

\begin{invoiceTable}

% \feetype{Consulting Services}

<% (loop for f in (@ fees) do %>
\hourrow
  {<%= fee-description f %>}
  {<%= princ-to-string (fee-quantity f) %>}
  {<%= format nil "~,2f" (/ (fee-price f) 100) %>}
<% ) %>

\subtotal

\end{invoiceTable}

\end{document}